\documentclass[journal,12pt,twocolumn]{IEEEtran}

\usepackage{setspace}
\usepackage{gensymb}

\singlespacing


\usepackage[cmex10]{amsmath}

\usepackage{amsthm}

\usepackage{mathrsfs}
\usepackage{txfonts}
\usepackage{stfloats}
\usepackage{bm}
\usepackage{cite}
\usepackage{cases}
\usepackage{subfig}

\usepackage{longtable}
\usepackage{multirow}

\usepackage{enumitem}
\usepackage{mathtools}
\usepackage{steinmetz}
\usepackage{tikz}
\usepackage{circuitikz}
\usepackage{verbatim}
\usepackage{tfrupee}
\usepackage[breaklinks=true]{hyperref}
\usepackage{graphicx}
\usepackage{tkz-euclide}

\usetikzlibrary{calc,math}
\usepackage{listings}
    \usepackage{color}                                            %%
    \usepackage{array}                                            %%
    \usepackage{longtable}                                        %%
    \usepackage{calc}                                             %%
    \usepackage{multirow}                                         %%
    \usepackage{hhline}                                           %%
    \usepackage{ifthen}                                           %%
    \usepackage{lscape}     
\usepackage{multicol}
\usepackage{chngcntr}

\DeclareMathOperator*{\Res}{Res}

\renewcommand\thesection{\arabic{section}}
\renewcommand\thesubsection{\thesection.\arabic{subsection}}
\renewcommand\thesubsubsection{\thesubsection.\arabic{subsubsection}}

\renewcommand\thesectiondis{\arabic{section}}
\renewcommand\thesubsectiondis{\thesectiondis.\arabic{subsection}}
\renewcommand\thesubsubsectiondis{\thesubsectiondis.\arabic{subsubsection}}


\hyphenation{op-tical net-works semi-conduc-tor}
\def\inputGnumericTable{}                                 %%

\lstset{
%language=C,
frame=single, 
breaklines=true,
columns=fullflexible
}
\begin{document}


\newtheorem{theorem}{Theorem}[section]
\newtheorem{problem}{Problem}
\newtheorem{proposition}{Proposition}[section]
\newtheorem{lemma}{Lemma}[section]
\newtheorem{corollary}[theorem]{Corollary}
\newtheorem{example}{Example}[section]
\newtheorem{definition}[problem]{Definition}

\newcommand{\BEQA}{\begin{eqnarray}}
\newcommand{\EEQA}{\end{eqnarray}}
\newcommand{\define}{\stackrel{\triangle}{=}}
\bibliographystyle{IEEEtran}
\providecommand{\mbf}{\mathbf}
\providecommand{\pr}[1]{\ensuremath{\Pr\left(#1\right)}}
\providecommand{\qfunc}[1]{\ensuremath{Q\left(#1\right)}}
\providecommand{\sbrak}[1]{\ensuremath{{}\left[#1\right]}}
\providecommand{\lsbrak}[1]{\ensuremath{{}\left[#1\right.}}
\providecommand{\rsbrak}[1]{\ensuremath{{}\left.#1\right]}}
\providecommand{\brak}[1]{\ensuremath{\left(#1\right)}}
\providecommand{\lbrak}[1]{\ensuremath{\left(#1\right.}}
\providecommand{\rbrak}[1]{\ensuremath{\left.#1\right)}}
\providecommand{\cbrak}[1]{\ensuremath{\left\{#1\right\}}}
\providecommand{\lcbrak}[1]{\ensuremath{\left\{#1\right.}}
\providecommand{\rcbrak}[1]{\ensuremath{\left.#1\right\}}}
\theoremstyle{remark}
\newtheorem{rem}{Remark}
\newcommand{\sgn}{\mathop{\mathrm{sgn}}}
\providecommand{\abs}[1]{\left\vert#1\right\vert}
\providecommand{\res}[1]{\Res\displaylimits_{#1}} 
\providecommand{\norm}[1]{\left\lVert#1\right\rVert}
%\providecommand{\norm}[1]{\lVert#1\rVert}
\providecommand{\mtx}[1]{\mathbf{#1}}
\providecommand{\mean}[1]{E\left[ #1 \right]}
\providecommand{\fourier}{\overset{\mathcal{F}}{ \rightleftharpoons}}
%\providecommand{\hilbert}{\overset{\mathcal{H}}{ \rightleftharpoons}}
\providecommand{\system}{\overset{\mathcal{H}}{ \longleftrightarrow}}
	%\newcommand{\solution}[2]{\textbf{Solution:}{#1}}
\newcommand{\solution}{\noindent \textbf{Solution: }}
\newcommand{\cosec}{\,\text{cosec}\,}
\providecommand{\dec}[2]{\ensuremath{\overset{#1}{\underset{#2}{\gtrless}}}}
\newcommand{\myvec}[1]{\ensuremath{\begin{pmatrix}#1\end{pmatrix}}}
\newcommand{\mydet}[1]{\ensuremath{\begin{vmatrix}#1\end{vmatrix}}}
\numberwithin{equation}{subsection}
\makeatletter
\@addtoreset{figure}{problem}
\makeatother
\let\StandardTheFigure\thefigure
\let\vec\mathbf
\renewcommand{\thefigure}{\theproblem}
\def\putbox#1#2#3{\makebox[0in][l]{\makebox[#1][l]{}\raisebox{\baselineskip}[0in][0in]{\raisebox{#2}[0in][0in]{#3}}}}
     \def\rightbox#1{\makebox[0in][r]{#1}}
     \def\centbox#1{\makebox[0in]{#1}}
     \def\topbox#1{\raisebox{-\baselineskip}[0in][0in]{#1}}
     \def\midbox#1{\raisebox{-0.5\baselineskip}[0in][0in]{#1}}
\vspace{3cm}
\title{Assignment-19}
\author{Ankur Aditya - EE20RESCH11010}
\maketitle
\newpage
\bigskip
\renewcommand{\thefigure}{\theenumi}
\renewcommand{\thetable}{\theenumi}

\begin{abstract}
This document contains the problem related to Elementary canonical forms (Hoffman Page-206, Section 6.4, Q-13) 
\end{abstract}
Download the latex-file from 
\begin{lstlisting}
https://github.com/ankuraditya13/EE5609-Assignment19
\end{lstlisting}

\section{Problem}
Let $\vec{V}$ be the space of n$\times$ n matrices over $\vec{F}$. Let $\vec{A}$ be a fixed n$\times$ n matrix over $\vec{F}$. Let T and U be the linear operators on $\vec{V}$ defined by 
\begin{align}
T(\vec{B}) = \vec{AB} \label{a}\\
U(\vec{B}) =\vec{AB} - \vec{BA}
\end{align}
\begin{enumerate}
\item[(a)] True of False? If A is diagonalizable (over $\vec{F}$), then T is diagonalizable. 
\item[(b)] True or False? If $\vec{A}$ is diagonalizable, then U is diagonalizable.
\end{enumerate}
\section{Theorems}
\subsection{Theorem 1}
Let $\lambda$ be a characteristic values of $\vec{T}$ and $\lambda \in \vec{F}$ and v is the corresponding characteristic vector which is a $n\times n$ matrix, then if $\vec{T}$ is a linear operator on finite dimensional space $\vec{V}$, it must be $\mydet{\vec{T}-\lambda \vec{I}} = 0$ and for $\brak{\vec{T}-\lambda \vec{I}}\vec{v}, \vec{v}\neq 0$.
\section{Solutions}
\begin{enumerate}
\item[(a)]Using theorem-1,
\begin{align}
\vec{Tv} = \lambda \vec{v}
\end{align}
Now, 
\begin{align}
\because \vec{Tv} = \vec{Av}\\
\implies \vec{Av} = \lambda \vec{v}\\
\implies \brak{\vec{A}-\lambda \vec{I}} \vec{v} = 0
\end{align}
Hence from above equation it is evident that $\vec{A}$ has the same characteristic values $\lambda$ as $\vec{T}$. Hence it can be concluded that if $\vec{A}$ is diagonalizable (over $\vec{F}$) then $\vec{T}$ is diagonalizable, as $\vec{A}$ is similar to $\vec{T}$ .
\item[(b)] Using Theorem-1,
\begin{align}
\vec{UB} = \lambda_1 \vec{B}
\end{align} 
\begin{align}
\implies \vec{AB-BA} = \lambda_1 \vec{B}\\
\implies \brak{\vec{A}-\lambda_1 \vec{I}} \vec{B} = \vec{BA}
\end{align}
Hence clearly $\vec{A}-\lambda_1 \vec{I}$ is not in null-space as,
\begin{align}
\mydet{\vec{A}-\lambda_1 \vec{I}} \neq 0
\end{align}
Hence, $\vec{U}$ and $\vec{T}$ are not similar, hence statement that if $\vec{A}$ is diagonalizable then $\vec{U}$ is diagonalizable is \textbf{False}
\end{enumerate}
\end{document}